\begin{center}
  \Large
  \textbf{KATA PENGANTAR}
\end{center}

\addcontentsline{toc}{chapter}{KATA PENGANTAR}

\vspace{2ex}

% Ubah paragraf-paragraf berikut dengan isi dari kata pengantar

Puji syukur kami panjatkan kehadirat Ida Sang Hyang Widhi Wasa  sehingga penelitian ini dapat disusun dan diselesaikan dengan baik. Dalam proses penyusunan tugas akhir ini, banyak pihak yang telah memberikan dukungan, bimbingan, dan motivasi kepada penulis. Oleh karena itu, dengan segala hormat dan rasa terima kasih  penulis menyampaikan penghargaan kepada:

\begin{enumerate}[nolistsep]
  \item Keluarga, Ibu Ni Nengah Sudini, Bapak I Nyoman Subina dan Saudara tercinta yang telah memberikan doa dan dukungan baik secara moral maupun material.
  \item Bapak Muhtadin, S.T, M.T dan Bapak Ahmad Zaini, S.T, M.T, selaku dosen pembimbing dari Teknik Komputer yang telah memberikan bimbingan, arahan, serta motivasi kepada penulis dalam menyelesaikan tugas akhir ini.
  \item Bapak Dr. Rudy Dikairono, S.T, M.T, selaku dosen pembimbing dari Tim Barunastra yang telah memberikan bimbingan, arahan, serta motivasi kepada penulis dalam menyelesaikan tugas akhir ini.
  \item Teman teman seperjuangan Gustu Dharma, Deva Febriana, Agus Ginting, Zein Bachtiar, Rifki Qolby, Jeremy Jhonson, Ikhsan Moekhtar, 
  \item Seorang wanita yang pernah hadir dalam hidup penulis, yang namanya tak dapat disebutkan, namun kehadirannya telah memberikan motivasi, menemani berbagai proses, dan membuat perjalanan penyusunan tugas akhir ini menjadi lebih bermakna



\end{enumerate}

Penulis menyadari bahwa penelitian ini masih jauh dari sempurna. Oleh karena itu, segala kritik dan saran yang membangun akan diterima dengan penuh keterbukaan demi perbaikan ke depannya.  Demikian, semoga penelitian ini dapat bermanfaat bagi perkembangan ilmu pengetahuan serta menjadi inspirasi bagi para pembaca.

\begin{flushright}
  \begin{tabular}[b]{c}
    \place{}, \MONTH{} \the\year{} \\
    \\
    \\
    \\
    \\
    \name{}
  \end{tabular}
\end{flushright}
